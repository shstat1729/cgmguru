\nonstopmode{}
\documentclass[a4paper]{book}
\usepackage[times,inconsolata,hyper]{Rd}
\usepackage{makeidx}
\makeatletter\@ifl@t@r\fmtversion{2018/04/01}{}{\usepackage[utf8]{inputenc}}\makeatother
% \usepackage{graphicx} % @USE GRAPHICX@
\makeindex{}
\begin{document}
\chapter*{}
\begin{center}
{\textbf{\huge Package `cgmguru'}}
\par\bigskip{\large \today}
\end{center}
\ifthenelse{\boolean{Rd@use@hyper}}{\hypersetup{pdftitle = {cgmguru: Continuous Glucose Monitoring Analysis and GRID-Based Event Detection}}}{}
\ifthenelse{\boolean{Rd@use@hyper}}{\hypersetup{pdfauthor = {Sang Ho Park}}}{}
\begin{description}
\raggedright{}
\item[Type]\AsIs{Package}
\item[Title]\AsIs{Continuous Glucose Monitoring Analysis and GRID-Based Event
Detection}
\item[Version]\AsIs{0.1.0}
\item[Description]\AsIs{Tools for analyzing Continuous Glucose Monitoring (CGM) time-series and
detecting glycemic events using GRID-based methodologies. Includes utilities for
identifying local maxima/minima, event segmentation around maxima, and summary
transformations for downstream analysis. Core algorithms are implemented in C++
via 'Rcpp' for performance.}
\item[License]\AsIs{MIT + file LICENSE}
\item[Encoding]\AsIs{UTF-8}
\item[RoxygenNote]\AsIs{7.3.2}
\item[LinkingTo]\AsIs{Rcpp}
\item[Imports]\AsIs{Rcpp}
\item[Suggests]\AsIs{testthat (>= 3.0.0), knitr, rmarkdown}
\item[VignetteBuilder]\AsIs{knitr}
\item[URL]\AsIs{}\url{https://github.com/shstat1729/cgmguru}\AsIs{}
\item[BugReports]\AsIs{}\url{https://github.com/shstat1729/cgmguru/issues}\AsIs{}
\item[NeedsCompilation]\AsIs{yes}
\item[Author]\AsIs{Sang Ho Park [aut, cre]}
\item[Maintainer]\AsIs{Sang Ho Park }\email{shstat1729@gmail.com}\AsIs{}
\end{description}
\Rdcontents{Contents}
\HeaderA{cgmguru-package}{cgmguru: Continuous Glucose Monitoring Analysis and GRID-Based Event Detection}{cgmguru.Rdash.package}
\aliasA{cgmguru}{cgmguru-package}{cgmguru}
\keyword{internal}{cgmguru-package}
%
\begin{Description}
Tools for analyzing Continuous Glucose Monitoring (CGM) time-series and detecting glycemic events using GRID-based methodologies. Includes utilities for identifying local maxima/minima, event segmentation around maxima, and summary transformations for downstream analysis. Core algorithms are implemented in C++ via 'Rcpp' for performance.
\end{Description}
%
\begin{Author}
\strong{Maintainer}: Sang Ho Park \email{shstat1729@gmail.com}

\end{Author}
%
\begin{SeeAlso}
Useful links:
\begin{itemize}

\item{} \url{https://github.com/shstat1729/cgmguru}
\item{} Report bugs at \url{https://github.com/shstat1729/cgmguru/issues}

\end{itemize}


\end{SeeAlso}
\HeaderA{orderfast}{Fast Ordering Function}{orderfast}
%
\begin{Description}
Orders a dataframe by id and time columns
\end{Description}
%
\begin{Usage}
\begin{verbatim}
orderfast(df)
\end{verbatim}
\end{Usage}
%
\begin{Arguments}
\begin{ldescription}
\item[\code{df}] A dataframe with 'id' and 'time' columns
\end{ldescription}
\end{Arguments}
%
\begin{Value}
A dataframe ordered by id and time
\end{Value}
%
\begin{Examples}
\begin{ExampleCode}
df <- data.frame(id = c("b", "a", "a"), time = as.POSIXct(
  c("2024-01-01 01:00:00", "2024-01-01 00:00:00", "2024-01-01 01:00:00"), tz = "UTC"
))
orderfast(df)
\end{ExampleCode}
\end{Examples}
\printindex{}
\end{document}
