\nonstopmode{}
\documentclass[a4paper]{book}
\usepackage[times,inconsolata,hyper]{Rd}
\usepackage{makeidx}
\makeatletter\@ifl@t@r\fmtversion{2018/04/01}{}{\usepackage[utf8]{inputenc}}\makeatother
% \usepackage{graphicx} % @USE GRAPHICX@
\makeindex{}
\begin{document}
\chapter*{}
\begin{center}
{\textbf{\huge Package `cgmguru'}}
\par\bigskip{\large \today}
\end{center}
\ifthenelse{\boolean{Rd@use@hyper}}{\hypersetup{pdftitle = {cgmguru: Advanced Continuous Glucose Monitoring Analysis with High-Performance C++ Backend}}}{}
\begin{description}
\raggedright{}
\item[Type]\AsIs{Package}
\item[Title]\AsIs{Advanced Continuous Glucose Monitoring Analysis with High-Performance C++ Backend}
\item[Version]\AsIs{0.1.0}
\item[Author]\AsIs{Sang Ho Park [aut, cre], Rosa Oh [aut, ctb], Sang-Man Jin [aut, ctb]}
\item[Maintainer]\AsIs{Sang Ho Park }\email{shstat1729@gmail.com}\AsIs{}
\item[Description]\AsIs{Tools for advanced analysis of continuous glucose monitoring (CGM)
time-series, implementing GRID (Glucose Rate Increase Detector) and GRID-based
algorithms for postprandial peak detection, and detection of hypoglycemic and
hyperglycemic episodes (Levels 1/2/Extended) aligned with international consensus
CGM metrics. Core algorithms are implemented in optimized C++ using 'Rcpp' to
provide accurate and fast analysis on large datasets.}
\item[License]\AsIs{MIT + file LICENSE}
\item[Encoding]\AsIs{UTF-8}
\item[RoxygenNote]\AsIs{7.3.3}
\item[LinkingTo]\AsIs{Rcpp}
\item[Imports]\AsIs{Rcpp}
\item[Suggests]\AsIs{testthat (>= 3.0.0),
knitr,
rmarkdown,
iglu,
dplyr,
covr,
ggplot2,
microbenchmark}
\item[VignetteBuilder]\AsIs{knitr}
\item[URL]\AsIs{}\url{https://github.com/shstat1729/cgmguru}\AsIs{}
\item[BugReports]\AsIs{}\url{https://github.com/shstat1729/cgmguru/issues}\AsIs{}
\item[Config/testthat/edition]\AsIs{3}
\end{description}
\Rdcontents{Contents}
\HeaderA{detect\_all\_events}{Detect All Glycemic Events}{detect.Rul.all.Rul.events}
%
\begin{Description}
Comprehensive function to detect all types of glycemic events aligned with 
international consensus CGM metrics (Battelino et al., 2023). This function 
provides a unified interface for detecting multiple event types including 
Level 1/2/Extended hypo- and hyperglycemia, and Level 1 excluded events.
\end{Description}
%
\begin{Usage}
\begin{verbatim}
detect_all_events(df, reading_minutes = NULL)
\end{verbatim}
\end{Usage}
%
\begin{Arguments}
\begin{ldescription}
\item[\code{df}] A dataframe containing continuous glucose monitoring (CGM) data.
Must include columns:
\begin{itemize}

\item{} \code{id}: Subject identifier (string or factor)
\item{} \code{time}: Time of measurement (POSIXct)
\item{} \code{gl}: Glucose value (integer or numeric, mg/dL)

\end{itemize}


\item[\code{reading\_minutes}] Time interval between readings in minutes (optional). Can be a single integer/numeric value (applied to all subjects) or a vector matching data length (different intervals per subject)
\end{ldescription}
\end{Arguments}
%
\begin{Value}
A tibble containing comprehensive event analysis with columns:
\begin{itemize}

\item{} \code{id}: Subject identifier
\item{} \code{type}: Event type (hypo/hyper)
\item{} \code{level}: Event level (lv1/lv2/extended/lv1\_excl)
\item{} \code{total\_episodes}: Total number of episodes
\item{} \code{avg\_ep\_per\_day}: Average episodes per day
\item{} \code{avg\_episode\_duration\_below\_54}: Average episode duration below 54 mg/dL in minutes (hypoglycemic events only)

\end{itemize}

\end{Value}
%
\begin{Section}{Event types}

- Hypoglycemia: lv1 (<70 mg/dL, >=15 min), lv2 (<54 mg/dL, >=15 min), extended (<70 mg/dL, >=120 min).
- Hyperglycemia: lv1 (>180 mg/dL, >=15 min), lv2 (>250 mg/dL, >=15 min), extended (>250 mg/dL, >=90 min in 120 min, end <=180 mg/dL for >=15 min).
\end{Section}
%
\begin{References}
Battelino, T., et al. (2023). Continuous glucose monitoring and metrics for clinical trials: an international consensus statement. The Lancet Diabetes \& Endocrinology, 11(1), 42-57.
\end{References}
%
\begin{SeeAlso}
\LinkA{detect\_hyperglycemic\_events}{detect.Rul.hyperglycemic.Rul.events}, \LinkA{detect\_hypoglycemic\_events}{detect.Rul.hypoglycemic.Rul.events}
\end{SeeAlso}
%
\begin{Examples}
\begin{ExampleCode}
# Load sample data
library(iglu)
data(example_data_5_subject)
data(example_data_hall)

# Detect all glycemic events with 5-minute reading intervals
all_events <- detect_all_events(example_data_5_subject, reading_minutes = 5)
print(all_events)

# Detect all events on larger dataset
large_all_events <- detect_all_events(example_data_hall, reading_minutes = 5)
print(paste("Total event types analyzed:", nrow(large_all_events)))

# Filter for specific event types
hyperglycemia_events <- all_events[all_events$type == "hyper", ]
hypoglycemia_events <- all_events[all_events$type == "hypo", ]

print("Hyperglycemia events:")
print(hyperglycemia_events)
print("Hypoglycemia events:")
print(hypoglycemia_events)
\end{ExampleCode}
\end{Examples}
\HeaderA{detect\_between\_maxima}{Detect Events Between Maxima}{detect.Rul.between.Rul.maxima}
\keyword{GRID pipeline}{detect\_between\_maxima}
%
\begin{Description}
Identifies and analyzes events occurring between detected maxima points,
providing detailed episode information for GRID analysis. This function
helps characterize the glucose dynamics between identified peaks.
\end{Description}
%
\begin{Usage}
\begin{verbatim}
detect_between_maxima(df, transform_df)
\end{verbatim}
\end{Usage}
%
\begin{Arguments}
\begin{ldescription}
\item[\code{df}] A dataframe containing continuous glucose monitoring (CGM) data.
Must include columns:
\begin{itemize}

\item{} \code{id}: Subject identifier (string or factor)
\item{} \code{time}: Time of measurement (POSIXct)
\item{} \code{gl}: Glucose value (integer or numeric, mg/dL)

\end{itemize}


\item[\code{transform\_df}] A dataframe containing summary information from previous transformations
\end{ldescription}
\end{Arguments}
%
\begin{Value}
A list containing:
\begin{itemize}

\item{} \code{results}: Tibble with events between maxima (\code{id}, \code{grid\_time}, \code{grid\_gl}, \code{maxima\_time}, \code{maxima\_glucose}, \code{time\_to\_peak})
\item{} \code{episode\_counts}: Tibble with episode counts per subject (\code{id}, \code{episode\_counts})

\end{itemize}

\end{Value}
%
\begin{SeeAlso}
\LinkA{grid}{grid}, \LinkA{mod\_grid}{mod.Rul.grid}, \LinkA{find\_new\_maxima}{find.Rul.new.Rul.maxima}, \LinkA{transform\_df}{transform.Rul.df}

Other GRID pipeline: 
\code{\LinkA{find\_local\_maxima}{find.Rul.local.Rul.maxima}()},
\code{\LinkA{find\_max\_after\_hours}{find.Rul.max.Rul.after.Rul.hours}()},
\code{\LinkA{find\_max\_before\_hours}{find.Rul.max.Rul.before.Rul.hours}()},
\code{\LinkA{find\_min\_after\_hours}{find.Rul.min.Rul.after.Rul.hours}()},
\code{\LinkA{find\_min\_before\_hours}{find.Rul.min.Rul.before.Rul.hours}()},
\code{\LinkA{find\_new\_maxima}{find.Rul.new.Rul.maxima}()},
\code{\LinkA{grid}{grid}()},
\code{\LinkA{maxima\_grid}{maxima.Rul.grid}()},
\code{\LinkA{mod\_grid}{mod.Rul.grid}()},
\code{\LinkA{start\_finder}{start.Rul.finder}()},
\code{\LinkA{transform\_df}{transform.Rul.df}()}
\end{SeeAlso}
%
\begin{Examples}
\begin{ExampleCode}
# Load sample data
library(iglu)
data(example_data_5_subject)
data(example_data_hall)

# Complete pipeline to get transform_df
grid_result <- grid(example_data_5_subject, gap = 60, threshold = 130)
maxima_result <- find_local_maxima(example_data_5_subject)
mod_result <- mod_grid(example_data_5_subject, grid_result$grid_vector, hours = 2, gap = 60)
max_after <- find_max_after_hours(example_data_5_subject, mod_result$mod_grid_vector, hours = 2)
new_maxima <- find_new_maxima(example_data_5_subject, 
                              max_after$max_indices, 
                              maxima_result$local_maxima_vector)
transformed <- transform_df(grid_result$episode_start, new_maxima)

# Detect events between maxima
between_events <- detect_between_maxima(example_data_5_subject, transformed)
print(paste("Events between maxima:", length(between_events)))

# Analysis on larger dataset
large_grid <- grid(example_data_hall, gap = 60, threshold = 130)
large_maxima <- find_local_maxima(example_data_hall)
large_mod <- mod_grid(example_data_hall, large_grid$grid_vector, hours = 2, gap = 60)
large_max_after <- find_max_after_hours(example_data_hall, large_mod$mod_grid_vector, hours = 2)
large_new_maxima <- find_new_maxima(example_data_hall, 
                                    large_max_after$max_indices, 
                                    large_maxima$local_maxima_vector)
large_transformed <- transform_df(large_grid$episode_start, large_new_maxima)
large_between <- detect_between_maxima(example_data_hall, large_transformed)
print(paste("Events between maxima in larger dataset:", length(large_between)))
\end{ExampleCode}
\end{Examples}
\HeaderA{detect\_hyperglycemic\_events}{Detect Hyperglycemic Events}{detect.Rul.hyperglycemic.Rul.events}
%
\begin{Description}
Identifies and segments hyperglycemic events in CGM data based on international consensus 
CGM metrics (Battelino et al., 2023). Supports three event types:
\begin{itemize}

\item{} \strong{Level 1}: >=15 consecutive min of >180 mg/dL, ends with >=15 consecutive min <=180 mg/dL
\item{} \strong{Level 2}: >=15 consecutive min of >250 mg/dL, ends with >=15 consecutive min <=250 mg/dL
\item{} \strong{Extended}: >250 mg/dL lasting >=90 cumulative min within a 120-min period, ends when glucose returns to <=180 mg/dL 
for >=15 consecutive min after

\end{itemize}

Events are detected when glucose exceeds the start threshold for the minimum duration and ends
when glucose falls below the end threshold for the specified end length.
\end{Description}
%
\begin{Usage}
\begin{verbatim}
detect_hyperglycemic_events(df,
 reading_minutes = NULL,
 dur_length = 120,
 end_length = 15,
 start_gl = 250,
 end_gl = 180)
\end{verbatim}
\end{Usage}
%
\begin{Arguments}
\begin{ldescription}
\item[\code{df}] A dataframe containing continuous glucose monitoring (CGM) data.
Must include columns:
\begin{itemize}

\item{} \code{id}: Subject identifier (string or factor)
\item{} \code{time}: Time of measurement (POSIXct)
\item{} \code{gl}: Glucose value (integer or numeric, mg/dL)

\end{itemize}


\item[\code{reading\_minutes}] Time interval between readings in minutes (optional)

\item[\code{dur\_length}] Minimum duration in minutes for event classification (default: 120)

\item[\code{end\_length}] End length criteria in minutes (default: 15)

\item[\code{start\_gl}] Starting glucose threshold in mg/dL (default: 250)

\item[\code{end\_gl}] Ending glucose threshold in mg/dL (default: 180)
\end{ldescription}
\end{Arguments}
%
\begin{Value}
A list containing:
\begin{itemize}

\item{} \code{events\_total}: Tibble with summary statistics per subject (id, total\_events, avg\_ep\_per\_day)
\item{} \code{events\_detailed}: Tibble with detailed event information (id, start\_time, start\_glucose, end\_time, end\_glucose, start\_indices, end\_indices)

\end{itemize}

\end{Value}
%
\begin{Section}{Units and sampling}

- \code{reading\_minutes} can be a scalar (all rows) or a vector per-row.
- If \code{reading\_minutes} is \code{NULL}, duration is computed from \code{time} deltas.
\end{Section}
%
\begin{References}
Battelino, T., et al. (2023). Continuous glucose monitoring and metrics for clinical trials: an international consensus statement. The Lancet Diabetes \& Endocrinology, 11(1), 42-57.
\end{References}
%
\begin{SeeAlso}
\LinkA{detect\_all\_events}{detect.Rul.all.Rul.events}
\end{SeeAlso}
%
\begin{Examples}
\begin{ExampleCode}
# Load sample data
library(iglu)
data(example_data_5_subject)
data(example_data_hall)

# Level 1 Hyperglycemia (>=15 consecutive min >180 mg/dL, ends <=180 mg/dL >=15 min)
hyper_lv1 <- detect_hyperglycemic_events(
  example_data_5_subject, 
  start_gl = 180, 
  dur_length = 15, 
  end_length = 15, 
  end_gl = 180
)
print(hyper_lv1$events_total)

# Level 2 Hyperglycemia (>=15 consecutive min >250 mg/dL, ends <=250 mg/dL >=15 min)
hyper_lv2 <- detect_hyperglycemic_events(
  example_data_5_subject, 
  start_gl = 250, 
  dur_length = 15, 
  end_length = 15, 
  end_gl = 250
)
print(hyper_lv2$events_total)

# Extended Hyperglycemia (>250 mg/dL >=90 cumulative min within 120-min period,
# ends <=180 mg/dL >=15 min after)
hyper_extended <- detect_hyperglycemic_events(example_data_5_subject)
print(hyper_extended$events_total)

# Compare event rates across levels
cat("Level 1 events:", sum(hyper_lv1$events_total$total_events), "\n")
cat("Level 2 events:", sum(hyper_lv2$events_total$total_events), "\n")
cat("Extended events:", sum(hyper_extended$events_total$total_events), "\n")

# Analysis on larger dataset with Level 1 criteria
large_hyper <- detect_hyperglycemic_events(example_data_hall, 
                                          start_gl = 180, 
                                          dur_length = 15, 
                                          end_length = 15, 
                                          end_gl = 180)
print(large_hyper$events_total)

# Analysis on larger dataset with Level 2 criteria
large_hyper_lv2 <- detect_hyperglycemic_events(example_data_hall,
                                               start_gl = 250,
                                               dur_length = 15,
                                               end_length = 15,
                                               end_gl = 250)
print(large_hyper_lv2$events_total)

# Analysis on larger dataset with Extended criteria
large_hyper_extended <- detect_hyperglycemic_events(example_data_hall)
print(large_hyper_extended$events_total)

# View detailed events for specific subject
if(nrow(hyper_lv1$events_detailed) > 0) {
  first_subject <- hyper_lv1$events_detailed$id[1]
  subject_events <- hyper_lv1$events_detailed[hyper_lv1$events_detailed$id == first_subject, ]
  head(subject_events)
}
\end{ExampleCode}
\end{Examples}
\HeaderA{detect\_hypoglycemic\_events}{Detect Hypoglycemic Events}{detect.Rul.hypoglycemic.Rul.events}
%
\begin{Description}
Identifies and segments hypoglycemic events in CGM data based on international consensus 
CGM metrics (Battelino et al., 2023). Supports three event types:
\begin{itemize}

\item{} \strong{Level 1}: >=15 consecutive min of <70 mg/dL, ends with >=15 consecutive min >=70 mg/dL
\item{} \strong{Level 2}: >=15 consecutive min of <54 mg/dL, ends with >=15 consecutive min >=54 mg/dL
\item{} \strong{Extended}: >120 consecutive min of <70 mg/dL, ends with >=15 consecutive min >=70 mg/dL

\end{itemize}

Events are detected when glucose falls below the start threshold for the minimum duration and ends
when glucose rises above the end threshold for the specified end length.
\end{Description}
%
\begin{Usage}
\begin{verbatim}
detect_hypoglycemic_events(df,
                                 reading_minutes = NULL,
                                 dur_length = 120,
                                 end_length = 15,
                                 start_gl = 70)
\end{verbatim}
\end{Usage}
%
\begin{Arguments}
\begin{ldescription}
\item[\code{df}] A dataframe containing continuous glucose monitoring (CGM) data.
Must include columns:
\begin{itemize}

\item{} \code{id}: Subject identifier (string or factor)
\item{} \code{time}: Time of measurement (POSIXct)
\item{} \code{gl}: Glucose value (integer or numeric, mg/dL)

\end{itemize}


\item[\code{reading\_minutes}] Time interval between readings in minutes (optional)

\item[\code{dur\_length}] Minimum duration in minutes for event classification (default: 120)

\item[\code{end\_length}] End length criteria in minutes (default: 15)

\item[\code{start\_gl}] Starting glucose threshold in mg/dL (default: 70)
\end{ldescription}
\end{Arguments}
%
\begin{Value}
A list containing:
\begin{itemize}

\item{} \code{events\_total}: Tibble with summary statistics per subject (id, total\_events, avg\_ep\_per\_day)
\item{} \code{events\_detailed}: Tibble with detailed event information (id, start\_time, start\_glucose, end\_time, end\_glucose, start\_indices, end\_indices, duration\_below\_54\_minutes)

\end{itemize}

\end{Value}
%
\begin{Section}{Units and sampling}

- \code{reading\_minutes} can be a scalar (all rows) or a vector per-row.
- If \code{reading\_minutes} is \code{NULL}, duration is computed from \code{time} deltas.
\end{Section}
%
\begin{References}
Battelino, T., et al. (2023). Continuous glucose monitoring and metrics for clinical trials: an international consensus statement. The Lancet Diabetes \& Endocrinology, 11(1), 42-57.
\end{References}
%
\begin{SeeAlso}
\LinkA{detect\_all\_events}{detect.Rul.all.Rul.events}
\end{SeeAlso}
%
\begin{Examples}
\begin{ExampleCode}
# Load sample data
library(iglu)
data(example_data_5_subject)
data(example_data_hall)

# Level 1 Hypoglycemia (<70 mg/dL >=15 consecutive min, ends >=70 mg/dL >=15 min)
hypo_lv1 <- detect_hypoglycemic_events(
  example_data_5_subject, 
  start_gl = 70, 
  dur_length = 15, 
  end_length = 15
)
print(hypo_lv1$events_total)

# Level 2 Hypoglycemia (<54 mg/dL >=15 consecutive min, ends >=54 mg/dL >=15 min)
hypo_lv2 <- detect_hypoglycemic_events(
  example_data_5_subject, 
  start_gl = 54, 
  dur_length = 15, 
  end_length = 15
)

# Extended Hypoglycemia (<70 mg/dL >=120 consecutive min, ends >=70 mg/dL >=15 min)
hypo_extended <- detect_hypoglycemic_events(example_data_5_subject)
print(hypo_extended$events_total)

# Compare event rates across levels
cat("Level 1 events:", sum(hypo_lv1$events_total$total_events), "\n")
cat("Level 2 events:", sum(hypo_lv2$events_total$total_events), "\n")
cat("Extended events:", sum(hypo_extended$events_total$total_events), "\n")

# Analysis on larger dataset with Level 1 criteria
large_hypo <- detect_hypoglycemic_events(example_data_hall, 
                                         start_gl = 70, 
                                         dur_length = 15, 
                                         end_length = 15)
print(large_hypo$events_total)

# Analysis on larger dataset with Level 2 criteria
large_hypo_lv2 <- detect_hypoglycemic_events(example_data_hall,
                                             start_gl = 54,
                                             dur_length = 15,
                                             end_length = 15)
print(large_hypo_lv2$events_total)

# Analysis on larger dataset with Extended criteria
large_hypo_extended <- detect_hypoglycemic_events(example_data_hall)
print(large_hypo_extended$events_total)
\end{ExampleCode}
\end{Examples}
\HeaderA{excursion}{Calculate Glucose Excursions}{excursion}
%
\begin{Description}
Calculates glucose excursions in CGM data. An excursion is defined as
a >70 mg/dL (>3.9 mmol/L) rise within 2 hours, not preceded by a value
<70 mg/dL (<3.9 mmol/L).
\end{Description}
%
\begin{Usage}
\begin{verbatim}
excursion(df, gap = 15)
\end{verbatim}
\end{Usage}
%
\begin{Arguments}
\begin{ldescription}
\item[\code{df}] A dataframe containing continuous glucose monitoring (CGM) data.
Must include columns:
\begin{itemize}

\item{} \code{id}: Subject identifier (string or factor)
\item{} \code{time}: Time of measurement (POSIXct)
\item{} \code{gl}: Glucose value (integer or numeric, mg/dL)

\end{itemize}


\item[\code{gap}] Gap threshold in minutes for excursion calculation (default: 15).
This parameter defines the minimum time interval between consecutive GRID events.
\end{ldescription}
\end{Arguments}
%
\begin{Value}
A list containing:
\begin{itemize}

\item{} \code{excursion\_vector}: Tibble with excursion results (\code{excursion})
\item{} \code{episode\_counts}: Tibble with episode counts per subject (\code{id}, \code{episode\_counts})
\item{} \code{episode\_start}: Tibble with all episode starts with columns:
\begin{itemize}

\item{} \code{id}: Subject identifier
\item{} \code{time}: Timestamp at which the event occurs; equivalent to \code{df\$time[indices]}
\item{} \code{gl}: Glucose value at the event; equivalent to \code{df\$gl[indices]}
\item{} \code{indices}: R-based (1-indexed) row number(s) in \code{df} denoting where the event occurs

\end{itemize}


\end{itemize}

\end{Value}
%
\begin{Section}{Notes}

- \code{gap} is minutes; change to enforce minimum separation between excursions.
\end{Section}
%
\begin{References}
Edwards, S., et al. (2022). Use of connected pen as a diagnostic tool to evaluate missed bolus dosing behavior in people with type 1 and type 2 diabetes. Diabetes Technology \& Therapeutics, 24(1), 61-66.
\end{References}
%
\begin{SeeAlso}
\LinkA{grid}{grid}
\end{SeeAlso}
%
\begin{Examples}
\begin{ExampleCode}
# Load sample data
library(iglu)
data(example_data_5_subject)
data(example_data_hall)

# Calculate glucose excursions
excursion_result <- excursion(example_data_5_subject, gap = 15)
print(paste("Excursion vector length:", length(excursion_result$excursion_vector)))
print(excursion_result$episode_counts)

# Excursion analysis with different gap
excursion_30min <- excursion(example_data_5_subject, gap = 30)

# Analysis on larger dataset
large_excursion <- excursion(example_data_hall, gap = 15)
print(paste("Excursion vector length in larger dataset:", length(large_excursion$excursion_vector)))
print(paste("Total episodes:", sum(large_excursion$episode_counts$episode_counts)))
\end{ExampleCode}
\end{Examples}
\HeaderA{find\_local\_maxima}{Find Local Maxima in Glucose Time Series}{find.Rul.local.Rul.maxima}
\keyword{GRID pipeline}{find\_local\_maxima}
%
\begin{Description}
Identifies local maxima (peaks) in glucose concentration time series data.
Uses a difference-based algorithm to detect peaks where glucose values
increase or remain constant for two consecutive points before the peak point,
and decrease or remain constant for two consecutive points after the peak point.
\end{Description}
%
\begin{Usage}
\begin{verbatim}
find_local_maxima(df)
\end{verbatim}
\end{Usage}
%
\begin{Arguments}
\begin{ldescription}
\item[\code{df}] A dataframe containing continuous glucose monitoring (CGM) data.
Must include columns:
\begin{itemize}

\item{} \code{id}: Subject identifier (string or factor)
\item{} \code{time}: Time of measurement (POSIXct)
\item{} \code{gl}: Glucose value (integer or numeric, mg/dL)

\end{itemize}

\end{ldescription}
\end{Arguments}
%
\begin{Value}
A list containing:
\begin{itemize}

\item{} \code{local\_maxima\_vector}: Tibble with R-based (1-indexed) row numbers of local maxima (\code{local\_maxima}). The corresponding occurrence time is \code{df\$time[local\_maxima]} and glucose is \code{df\$gl[local\_maxima]}.
\item{} \code{merged\_results}: Tibble with local maxima details (\code{id}, \code{time}, \code{gl})

\end{itemize}

\end{Value}
%
\begin{SeeAlso}
\LinkA{grid}{grid}, \LinkA{mod\_grid}{mod.Rul.grid}, \LinkA{find\_new\_maxima}{find.Rul.new.Rul.maxima}

Other GRID pipeline: 
\code{\LinkA{detect\_between\_maxima}{detect.Rul.between.Rul.maxima}()},
\code{\LinkA{find\_max\_after\_hours}{find.Rul.max.Rul.after.Rul.hours}()},
\code{\LinkA{find\_max\_before\_hours}{find.Rul.max.Rul.before.Rul.hours}()},
\code{\LinkA{find\_min\_after\_hours}{find.Rul.min.Rul.after.Rul.hours}()},
\code{\LinkA{find\_min\_before\_hours}{find.Rul.min.Rul.before.Rul.hours}()},
\code{\LinkA{find\_new\_maxima}{find.Rul.new.Rul.maxima}()},
\code{\LinkA{grid}{grid}()},
\code{\LinkA{maxima\_grid}{maxima.Rul.grid}()},
\code{\LinkA{mod\_grid}{mod.Rul.grid}()},
\code{\LinkA{start\_finder}{start.Rul.finder}()},
\code{\LinkA{transform\_df}{transform.Rul.df}()}
\end{SeeAlso}
%
\begin{Examples}
\begin{ExampleCode}
# Load sample data
library(iglu)
data(example_data_5_subject)
data(example_data_hall)

# Find local maxima
maxima_result <- find_local_maxima(example_data_5_subject)
print(paste("Found", nrow(maxima_result$local_maxima_vector), "local maxima"))

# Find maxima on larger dataset
large_maxima <- find_local_maxima(example_data_hall)
print(paste("Found", nrow(large_maxima$local_maxima_vector), "local maxima in larger dataset"))

# View first few maxima
head(maxima_result$local_maxima_vector)

# View merged results
head(maxima_result$merged_results)
\end{ExampleCode}
\end{Examples}
\HeaderA{find\_max\_after\_hours}{Find Maximum Glucose After Specified Hours}{find.Rul.max.Rul.after.Rul.hours}
\keyword{GRID pipeline}{find\_max\_after\_hours}
%
\begin{Description}
Identifies the maximum glucose value occurring within a specified time window
after a given start point. This function is useful for analyzing glucose
patterns following specific events or time points.
\end{Description}
%
\begin{Usage}
\begin{verbatim}
find_max_after_hours(df, start_point_df, hours)
\end{verbatim}
\end{Usage}
%
\begin{Arguments}
\begin{ldescription}
\item[\code{df}] A dataframe containing continuous glucose monitoring (CGM) data.
Must include columns:
\begin{itemize}

\item{} \code{id}: Subject identifier (string or factor)
\item{} \code{time}: Time of measurement (POSIXct)
\item{} \code{gl}: Glucose value (integer or numeric, mg/dL)

\end{itemize}


\item[\code{start\_point\_df}] A dataframe with column \code{start\_indices} (R-based indices into \code{df})

\item[\code{hours}] Number of hours to look ahead from the start point
\end{ldescription}
\end{Arguments}
%
\begin{Value}
A list containing:
\begin{itemize}

\item{} \code{max\_indices}: Tibble with R-based (1-indexed) row numbers of maximum glucose (\code{max\_indices}).
The corresponding occurrence time is \code{df\$time[max\_indices]} and glucose is \code{df\$gl[max\_indices]}.
\item{} \code{episode\_counts}: Tibble with episode counts per subject (\code{id}, \code{episode\_counts})
\item{} \code{episode\_start}: Tibble with all episode starts with columns:
\begin{itemize}

\item{} \code{id}: Subject identifier
\item{} \code{time}: Timestamp at which the maximum occurs; equivalent to \code{df\$time[indices]}
\item{} \code{gl}: Glucose value at the maximum; equivalent to \code{df\$gl[indices]}
\item{} \code{indices}: R-based (1-indexed) row number(s) in \code{df} denoting where the maximum occurs

\end{itemize}


\end{itemize}

\end{Value}
%
\begin{Section}{Notes}

- \code{start\_indices} must be valid row numbers in \code{df} (1-indexed).
- The search window is (0, \code{hours}] hours after each start index.
\end{Section}
%
\begin{SeeAlso}
\LinkA{mod\_grid}{mod.Rul.grid}, \LinkA{find\_local\_maxima}{find.Rul.local.Rul.maxima}, \LinkA{find\_new\_maxima}{find.Rul.new.Rul.maxima}, \LinkA{transform\_df}{transform.Rul.df}

Other GRID pipeline: 
\code{\LinkA{detect\_between\_maxima}{detect.Rul.between.Rul.maxima}()},
\code{\LinkA{find\_local\_maxima}{find.Rul.local.Rul.maxima}()},
\code{\LinkA{find\_max\_before\_hours}{find.Rul.max.Rul.before.Rul.hours}()},
\code{\LinkA{find\_min\_after\_hours}{find.Rul.min.Rul.after.Rul.hours}()},
\code{\LinkA{find\_min\_before\_hours}{find.Rul.min.Rul.before.Rul.hours}()},
\code{\LinkA{find\_new\_maxima}{find.Rul.new.Rul.maxima}()},
\code{\LinkA{grid}{grid}()},
\code{\LinkA{maxima\_grid}{maxima.Rul.grid}()},
\code{\LinkA{mod\_grid}{mod.Rul.grid}()},
\code{\LinkA{start\_finder}{start.Rul.finder}()},
\code{\LinkA{transform\_df}{transform.Rul.df}()}
\end{SeeAlso}
%
\begin{Examples}
\begin{ExampleCode}
# Load sample data
library(iglu)
data(example_data_5_subject)
data(example_data_hall)

# Create start points for demonstration (using row indices)
start_indices <- seq(1, nrow(example_data_5_subject), by = 100)
start_points <- data.frame(start_indices = start_indices)

# Find maximum glucose in next 2 hours
max_after <- find_max_after_hours(example_data_5_subject, start_points, hours = 2)
print(paste("Found", length(max_after$max_indices), "maximum points"))

# Find maximum glucose in next 1 hour
max_after_1h <- find_max_after_hours(example_data_5_subject, start_points, hours = 1)

# Analysis on larger dataset
large_start_indices <- seq(1, nrow(example_data_hall), by = 200)
large_start_points <- data.frame(start_indices = large_start_indices)
large_max_after <- find_max_after_hours(example_data_hall, large_start_points, hours = 2)
print(paste("Found", length(large_max_after$max_indices), "maximum points in larger dataset"))
\end{ExampleCode}
\end{Examples}
\HeaderA{find\_max\_before\_hours}{Find Maximum Glucose Before Specified Hours}{find.Rul.max.Rul.before.Rul.hours}
\keyword{GRID pipeline}{find\_max\_before\_hours}
%
\begin{Description}
Identifies the maximum glucose value occurring within a specified time window
before a given start point. This function is useful for analyzing glucose
patterns preceding specific events or time points.
\end{Description}
%
\begin{Usage}
\begin{verbatim}
find_max_before_hours(df, start_point_df, hours)
\end{verbatim}
\end{Usage}
%
\begin{Arguments}
\begin{ldescription}
\item[\code{df}] A dataframe containing continuous glucose monitoring (CGM) data.
Must include columns:
\begin{itemize}

\item{} \code{id}: Subject identifier (string or factor)
\item{} \code{time}: Time of measurement (POSIXct)
\item{} \code{gl}: Glucose value (integer or numeric, mg/dL)

\end{itemize}


\item[\code{start\_point\_df}] A dataframe with column \code{start\_indices} (R-based indices into \code{df})

\item[\code{hours}] Number of hours to look back from the start point
\end{ldescription}
\end{Arguments}
%
\begin{Value}
A list containing:
\begin{itemize}

\item{} \code{max\_indices}: Tibble with R-based (1-indexed) row numbers of maximum glucose (\code{max\_indices}).
The corresponding occurrence time is \code{df\$time[max\_indices]} and glucose is \code{df\$gl[max\_indices]}.
\item{} \code{episode\_counts}: Tibble with episode counts per subject (\code{id}, \code{episode\_counts})
\item{} \code{episode\_start}: Tibble with all episode starts with columns:
\begin{itemize}

\item{} \code{id}: Subject identifier
\item{} \code{time}: Timestamp at which the maximum occurs; equivalent to \code{df\$time[indices]}
\item{} \code{gl}: Glucose value at the maximum; equivalent to \code{df\$gl[indices]}
\item{} \code{indices}: R-based (1-indexed) row number(s) in \code{df} denoting where the maximum occurs

\end{itemize}


\end{itemize}

\end{Value}
%
\begin{Section}{Notes}

- The search window is [\code{hours}, 0) hours before each start index.
\end{Section}
%
\begin{SeeAlso}
\LinkA{mod\_grid}{mod.Rul.grid}, \LinkA{find\_local\_maxima}{find.Rul.local.Rul.maxima}, \LinkA{find\_new\_maxima}{find.Rul.new.Rul.maxima}

Other GRID pipeline: 
\code{\LinkA{detect\_between\_maxima}{detect.Rul.between.Rul.maxima}()},
\code{\LinkA{find\_local\_maxima}{find.Rul.local.Rul.maxima}()},
\code{\LinkA{find\_max\_after\_hours}{find.Rul.max.Rul.after.Rul.hours}()},
\code{\LinkA{find\_min\_after\_hours}{find.Rul.min.Rul.after.Rul.hours}()},
\code{\LinkA{find\_min\_before\_hours}{find.Rul.min.Rul.before.Rul.hours}()},
\code{\LinkA{find\_new\_maxima}{find.Rul.new.Rul.maxima}()},
\code{\LinkA{grid}{grid}()},
\code{\LinkA{maxima\_grid}{maxima.Rul.grid}()},
\code{\LinkA{mod\_grid}{mod.Rul.grid}()},
\code{\LinkA{start\_finder}{start.Rul.finder}()},
\code{\LinkA{transform\_df}{transform.Rul.df}()}
\end{SeeAlso}
%
\begin{Examples}
\begin{ExampleCode}
# Load sample data
library(iglu)
data(example_data_5_subject)
data(example_data_hall)

# Create start points for demonstration (using row indices)
start_indices <- seq(1, nrow(example_data_5_subject), by = 100)
start_points <- data.frame(start_indices = start_indices)

# Find maximum glucose in previous 2 hours
max_before <- find_max_before_hours(example_data_5_subject, start_points, hours = 2)
print(paste("Found", length(max_before$max_indices), "maximum points"))

# Find maximum glucose in previous 1 hour
max_before_1h <- find_max_before_hours(example_data_5_subject, start_points, hours = 1)

# Analysis on larger dataset
large_start_indices <- seq(1, nrow(example_data_hall), by = 200)
large_start_points <- data.frame(start_indices = large_start_indices)
large_max_before <- find_max_before_hours(example_data_hall, large_start_points, hours = 2)
print(paste("Found", length(large_max_before$max_indices), "maximum points in larger dataset"))
\end{ExampleCode}
\end{Examples}
\HeaderA{find\_min\_after\_hours}{Find Minimum Glucose After Specified Hours}{find.Rul.min.Rul.after.Rul.hours}
\keyword{GRID pipeline}{find\_min\_after\_hours}
%
\begin{Description}
Identifies the minimum glucose value occurring within a specified time window
after a given start point. This function is useful for analyzing glucose
patterns following specific events or time points.
\end{Description}
%
\begin{Usage}
\begin{verbatim}
find_min_after_hours(df, start_point_df, hours)
\end{verbatim}
\end{Usage}
%
\begin{Arguments}
\begin{ldescription}
\item[\code{df}] A dataframe containing continuous glucose monitoring (CGM) data.
Must include columns:
\begin{itemize}

\item{} \code{id}: Subject identifier (string or factor)
\item{} \code{time}: Time of measurement (POSIXct)
\item{} \code{gl}: Glucose value (integer or numeric, mg/dL)

\end{itemize}


\item[\code{start\_point\_df}] A dataframe with column \code{start\_indices} (R-based indices into \code{df})

\item[\code{hours}] Number of hours to look ahead from the start point
\end{ldescription}
\end{Arguments}
%
\begin{Value}
A list containing:
\begin{itemize}

\item{} \code{min\_indices}: Tibble with R-based (1-indexed) row numbers of minimum glucose (\code{min\_indices}). The corresponding occurrence time is \code{df\$time[min\_indices]} and glucose is \code{df\$gl[min\_indices]}.
\item{} \code{episode\_counts}: Tibble with episode counts per subject (\code{id}, \code{episode\_counts})
\item{} \code{episode\_start}: Tibble with all episode starts with columns:
\begin{itemize}

\item{} \code{id}: Subject identifier
\item{} \code{time}: Timestamp at which the minimum occurs; equivalent to \code{df\$time[indices]}
\item{} \code{gl}: Glucose value at the minimum; equivalent to \code{df\$gl[indices]}
\item{} \code{indices}: R-based (1-indexed) row number(s) in \code{df} denoting where the minimum occurs

\end{itemize}


\end{itemize}

\end{Value}
%
\begin{SeeAlso}
\LinkA{mod\_grid}{mod.Rul.grid}, \LinkA{find\_local\_maxima}{find.Rul.local.Rul.maxima}

Other GRID pipeline: 
\code{\LinkA{detect\_between\_maxima}{detect.Rul.between.Rul.maxima}()},
\code{\LinkA{find\_local\_maxima}{find.Rul.local.Rul.maxima}()},
\code{\LinkA{find\_max\_after\_hours}{find.Rul.max.Rul.after.Rul.hours}()},
\code{\LinkA{find\_max\_before\_hours}{find.Rul.max.Rul.before.Rul.hours}()},
\code{\LinkA{find\_min\_before\_hours}{find.Rul.min.Rul.before.Rul.hours}()},
\code{\LinkA{find\_new\_maxima}{find.Rul.new.Rul.maxima}()},
\code{\LinkA{grid}{grid}()},
\code{\LinkA{maxima\_grid}{maxima.Rul.grid}()},
\code{\LinkA{mod\_grid}{mod.Rul.grid}()},
\code{\LinkA{start\_finder}{start.Rul.finder}()},
\code{\LinkA{transform\_df}{transform.Rul.df}()}
\end{SeeAlso}
%
\begin{Examples}
\begin{ExampleCode}
# Load sample data
library(iglu)
data(example_data_5_subject)
data(example_data_hall)

# Create start points for demonstration (using row indices)
start_indices <- seq(1, nrow(example_data_5_subject), by = 100)
start_points <- data.frame(start_indices = start_indices)

# Find minimum glucose in next 2 hours
min_after <- find_min_after_hours(example_data_5_subject, start_points, hours = 2)
print(paste("Found", length(min_after$min_indices), "minimum points"))

# Find minimum glucose in next 1 hour
min_after_1h <- find_min_after_hours(example_data_5_subject, start_points, hours = 1)

# Analysis on larger dataset
large_start_indices <- seq(1, nrow(example_data_hall), by = 200)
large_start_points <- data.frame(start_indices = large_start_indices)
large_min_after <- find_min_after_hours(example_data_hall, large_start_points, hours = 2)
print(paste("Found", length(large_min_after$min_indices), "minimum points in larger dataset"))
\end{ExampleCode}
\end{Examples}
\HeaderA{find\_min\_before\_hours}{Find Minimum Glucose Before Specified Hours}{find.Rul.min.Rul.before.Rul.hours}
\keyword{GRID pipeline}{find\_min\_before\_hours}
%
\begin{Description}
Identifies the minimum glucose value occurring within a specified time window
before a given start point. This function is useful for analyzing glucose
patterns preceding specific events or time points.
\end{Description}
%
\begin{Usage}
\begin{verbatim}
find_min_before_hours(df, start_point_df, hours)
\end{verbatim}
\end{Usage}
%
\begin{Arguments}
\begin{ldescription}
\item[\code{df}] A dataframe containing continuous glucose monitoring (CGM) data.
Must include columns:
\begin{itemize}

\item{} \code{id}: Subject identifier (string or factor)
\item{} \code{time}: Time of measurement (POSIXct)
\item{} \code{gl}: Glucose value (integer or numeric, mg/dL)

\end{itemize}


\item[\code{start\_point\_df}] A dataframe with column \code{start\_indices} (R-based indices into \code{df})

\item[\code{hours}] Number of hours to look back from the start point
\end{ldescription}
\end{Arguments}
%
\begin{Value}
A list containing:
\begin{itemize}

\item{} \code{min\_indices}: Tibble with R-based (1-indexed) row numbers of minimum glucose (\code{min\_indices}). The corresponding occurrence time is \code{df\$time[min\_indices]} and glucose is \code{df\$gl[min\_indices]}.
\item{} \code{episode\_counts}: Tibble with episode counts per subject (\code{id}, \code{episode\_counts})
\item{} \code{episode\_start}: Tibble with all episode starts with columns:
\begin{itemize}

\item{} \code{id}: Subject identifier
\item{} \code{time}: Timestamp at which the minimum occurs; equivalent to \code{df\$time[indices]}
\item{} \code{gl}: Glucose value at the minimum; equivalent to \code{df\$gl[indices]}
\item{} \code{indices}: R-based (1-indexed) row number(s) in \code{df} denoting where the minimum occurs

\end{itemize}


\end{itemize}

\end{Value}
%
\begin{SeeAlso}
\LinkA{mod\_grid}{mod.Rul.grid}, \LinkA{find\_local\_maxima}{find.Rul.local.Rul.maxima}

Other GRID pipeline: 
\code{\LinkA{detect\_between\_maxima}{detect.Rul.between.Rul.maxima}()},
\code{\LinkA{find\_local\_maxima}{find.Rul.local.Rul.maxima}()},
\code{\LinkA{find\_max\_after\_hours}{find.Rul.max.Rul.after.Rul.hours}()},
\code{\LinkA{find\_max\_before\_hours}{find.Rul.max.Rul.before.Rul.hours}()},
\code{\LinkA{find\_min\_after\_hours}{find.Rul.min.Rul.after.Rul.hours}()},
\code{\LinkA{find\_new\_maxima}{find.Rul.new.Rul.maxima}()},
\code{\LinkA{grid}{grid}()},
\code{\LinkA{maxima\_grid}{maxima.Rul.grid}()},
\code{\LinkA{mod\_grid}{mod.Rul.grid}()},
\code{\LinkA{start\_finder}{start.Rul.finder}()},
\code{\LinkA{transform\_df}{transform.Rul.df}()}
\end{SeeAlso}
%
\begin{Examples}
\begin{ExampleCode}
# Load sample data
library(iglu)
data(example_data_5_subject)
data(example_data_hall)

# Create start points for demonstration (using row indices)
start_indices <- seq(1, nrow(example_data_5_subject), by = 100)
start_points <- data.frame(start_indices = start_indices)

# Find minimum glucose in previous 2 hours
min_before <- find_min_before_hours(example_data_5_subject, start_points, hours = 2)
print(paste("Found", length(min_before$min_indices), "minimum points"))

# Find minimum glucose in previous 1 hour
min_before_1h <- find_min_before_hours(example_data_5_subject, start_points, hours = 1)

# Analysis on larger dataset
large_start_indices <- seq(1, nrow(example_data_hall), by = 200)
large_start_points <- data.frame(start_indices = large_start_indices)
large_min_before <- find_min_before_hours(example_data_hall, large_start_points, hours = 2)
print(paste("Found", length(large_min_before$min_indices), "minimum points in larger dataset"))
\end{ExampleCode}
\end{Examples}
\HeaderA{find\_new\_maxima}{Find New Maxima Around Grid Points}{find.Rul.new.Rul.maxima}
\keyword{GRID pipeline}{find\_new\_maxima}
%
\begin{Description}
Identifies new maxima in the vicinity of previously identified grid points,
useful for refining maxima detection in GRID analysis. This function helps
improve the accuracy of peak detection by searching around known event points.
\end{Description}
%
\begin{Usage}
\begin{verbatim}
find_new_maxima(df, mod_grid_max_point_df, local_maxima_df)
\end{verbatim}
\end{Usage}
%
\begin{Arguments}
\begin{ldescription}
\item[\code{df}] A dataframe containing continuous glucose monitoring (CGM) data.
Must include columns:
\begin{itemize}

\item{} \code{id}: Subject identifier (string or factor)
\item{} \code{time}: Time of measurement (POSIXct)
\item{} \code{gl}: Glucose value (integer or numeric, mg/dL)

\end{itemize}


\item[\code{mod\_grid\_max\_point\_df}] A dataframe with column \code{indices} (candidate maxima indices)

\item[\code{local\_maxima\_df}] A dataframe with column \code{local\_maxima} (indices of local peaks)
\end{ldescription}
\end{Arguments}
%
\begin{Value}
A tibble with updated maxima information containing columns (\code{id}, \code{time}, \code{gl}, \code{indices})
The \code{indices} column contains R-based (1-indexed) row number(s) in \code{df}; thus, \code{time == df\$time[indices]} and \code{gl == df\$gl[indices]}.
\end{Value}
%
\begin{SeeAlso}
\LinkA{find\_local\_maxima}{find.Rul.local.Rul.maxima}, \LinkA{find\_max\_after\_hours}{find.Rul.max.Rul.after.Rul.hours}, \LinkA{transform\_df}{transform.Rul.df}

Other GRID pipeline: 
\code{\LinkA{detect\_between\_maxima}{detect.Rul.between.Rul.maxima}()},
\code{\LinkA{find\_local\_maxima}{find.Rul.local.Rul.maxima}()},
\code{\LinkA{find\_max\_after\_hours}{find.Rul.max.Rul.after.Rul.hours}()},
\code{\LinkA{find\_max\_before\_hours}{find.Rul.max.Rul.before.Rul.hours}()},
\code{\LinkA{find\_min\_after\_hours}{find.Rul.min.Rul.after.Rul.hours}()},
\code{\LinkA{find\_min\_before\_hours}{find.Rul.min.Rul.before.Rul.hours}()},
\code{\LinkA{grid}{grid}()},
\code{\LinkA{maxima\_grid}{maxima.Rul.grid}()},
\code{\LinkA{mod\_grid}{mod.Rul.grid}()},
\code{\LinkA{start\_finder}{start.Rul.finder}()},
\code{\LinkA{transform\_df}{transform.Rul.df}()}
\end{SeeAlso}
%
\begin{Examples}
\begin{ExampleCode}
# Load sample data
library(iglu)
data(example_data_5_subject)
data(example_data_hall)

# First, get grid points and local maxima
grid_result <- grid(example_data_5_subject, gap = 15, threshold = 130)
maxima_result <- find_local_maxima(example_data_5_subject)

# Create modified grid points (simplified for example)
mod_grid_indices <- data.frame(indices = grid_result$episode_start$indices[1:10])

# Find new maxima around grid points
new_maxima <- find_new_maxima(example_data_5_subject, 
                              mod_grid_indices, 
                              maxima_result$local_maxima_vector)
print(paste("Found", nrow(new_maxima), "new maxima"))

# Analysis on larger dataset
large_grid <- grid(example_data_hall, gap = 15, threshold = 130)
large_maxima <- find_local_maxima(example_data_hall)
large_mod_grid <- data.frame(indices = large_grid$episode_start$indices[1:20])
large_new_maxima <- find_new_maxima(example_data_hall, 
                                    large_mod_grid, 
                                    large_maxima$local_maxima_vector)
print(paste("Found", nrow(large_new_maxima), "new maxima in larger dataset"))
\end{ExampleCode}
\end{Examples}
\HeaderA{grid}{GRID Algorithm for Glycemic Event Detection}{grid}
\keyword{GRID pipeline}{grid}
%
\begin{Description}
Implements the GRID (Glucose Rate Increase Detector) algorithm for detecting rapid glucose rate increases in continuous glucose monitoring (CGM) data.
This algorithm identifies rapid glucose changes using specific rate-based criteria, and is commonly applied for meal detection.
Meals are detected when the CGM value is >=7.2 mmol/L (>=130 mg/dL) and the rate-of-change is >=5.3 mmol/L/h [>=95 mg/dL/h] for the last two consecutive readings, or >=5.0 mmol/L/h [>=90 mg/dL/h] for two of the last three readings.
\end{Description}
%
\begin{Usage}
\begin{verbatim}
grid(df, gap = 15, threshold = 130)
\end{verbatim}
\end{Usage}
%
\begin{Arguments}
\begin{ldescription}
\item[\code{df}] A dataframe containing continuous glucose monitoring (CGM) data.
Must include columns:
\begin{itemize}

\item{} \code{id}: Subject identifier (string or factor)
\item{} \code{time}: Time of measurement (POSIXct)
\item{} \code{gl}: Glucose value (integer or numeric, mg/dL)

\end{itemize}


\item[\code{gap}] Gap threshold in minutes for event detection (default: 15).
This parameter defines the minimum time interval between consecutive GRID events. For example, if gap is set to 60, only one GRID event can be detected within any one-hour window; subsequent events within the gap interval are not counted as new events.

\item[\code{threshold}] GRID slope threshold in mg/dL/hour for event classification (default: 130)
\end{ldescription}
\end{Arguments}
%
\begin{Value}
A list containing:
\begin{itemize}

\item{} \code{grid\_vector}: A tibble with the results of the GRID analysis. Contains a \code{grid} column (0/1 values; 1 denotes a detected GRID event) and all relevant input columns.
\item{} \code{episode\_counts}: A tibble summarizing the number of GRID events per subject (\code{id}) as \code{episode\_counts}.
\item{} \code{episode\_start}: A tibble listing the start of each GRID episode, with columns:
\begin{itemize}

\item{} \code{id}: Subject ID.
\item{} \code{time}: The timestamp (POSIXct) at which the GRID event was detected.
\item{} \code{gl}: The glucose value (mg/dL; integer or numeric) at the GRID event.
\item{} \code{indices}: R-based (1-indexed) row number(s) in the original dataframe where the GRID event occurs. The occurrence time equals \code{df\$time[indices]} and glucose equals \code{df\$gl[indices]}.

\end{itemize}


\end{itemize}

\end{Value}
%
\begin{Section}{Algorithm}

- Flags points where \code{gl >= 130 mg/dL} and rate-of-change meets the GRID criteria (see references).
- Enforces a minimum \code{gap} in minutes between detected events to avoid duplicates.
\end{Section}
%
\begin{Section}{Units and sampling}

- \code{gl} is mg/dL; \code{time} is POSIXct; \code{gap} is minutes.
- The effective sampling interval is derived from \code{time} deltas.
\end{Section}
%
\begin{References}
Harvey, R. A., et al. (2014). Design of the glucose rate increase detector: a meal detection module for the health monitoring system. Journal of Diabetes Science and Technology, 8(2), 307-320.

Adolfsson, Peter, et al. "Increased time in range and fewer missed bolus injections after introduction of a smart connected insulin pen." Diabetes technology \& therapeutics 22.10 (2020): 709-718.
\end{References}
%
\begin{SeeAlso}
\LinkA{mod\_grid}{mod.Rul.grid}, \LinkA{maxima\_grid}{maxima.Rul.grid}, \LinkA{find\_local\_maxima}{find.Rul.local.Rul.maxima}, \LinkA{detect\_between\_maxima}{detect.Rul.between.Rul.maxima}

Other GRID pipeline: 
\code{\LinkA{detect\_between\_maxima}{detect.Rul.between.Rul.maxima}()},
\code{\LinkA{find\_local\_maxima}{find.Rul.local.Rul.maxima}()},
\code{\LinkA{find\_max\_after\_hours}{find.Rul.max.Rul.after.Rul.hours}()},
\code{\LinkA{find\_max\_before\_hours}{find.Rul.max.Rul.before.Rul.hours}()},
\code{\LinkA{find\_min\_after\_hours}{find.Rul.min.Rul.after.Rul.hours}()},
\code{\LinkA{find\_min\_before\_hours}{find.Rul.min.Rul.before.Rul.hours}()},
\code{\LinkA{find\_new\_maxima}{find.Rul.new.Rul.maxima}()},
\code{\LinkA{maxima\_grid}{maxima.Rul.grid}()},
\code{\LinkA{mod\_grid}{mod.Rul.grid}()},
\code{\LinkA{start\_finder}{start.Rul.finder}()},
\code{\LinkA{transform\_df}{transform.Rul.df}()}
\end{SeeAlso}
%
\begin{Examples}
\begin{ExampleCode}
# Load sample data
library(iglu)
data(example_data_5_subject)
data(example_data_hall)

# Basic GRID analysis on smaller dataset
grid_result <- grid(example_data_5_subject, gap = 15, threshold = 130)
print(grid_result$episode_counts)
print(grid_result$episode_start)
print(grid_result$grid_vector)

# More sensitive GRID analysis
sensitive_result <- grid(example_data_5_subject, gap = 10, threshold = 120)

# GRID analysis on larger dataset
large_grid <- grid(example_data_hall, gap = 15, threshold = 130)
print(paste("Detected", sum(large_grid$episode_counts$episode_counts), "episodes"))
print(large_grid$episode_start)
print(large_grid$grid_vector)

\end{ExampleCode}
\end{Examples}
\HeaderA{maxima\_grid}{Combined Maxima Detection and GRID Analysis}{maxima.Rul.grid}
\keyword{GRID pipeline}{maxima\_grid}
%
\begin{Description}
Fast method for postprandial glucose peak detection combining GRID algorithm with local maxima analysis.
Detects meal-induced glucose peaks by identifying GRID events (rapid glucose increases) and mapping
them to corresponding local maxima within a search window. Local maxima are defined as points where
glucose values increase or remain constant for two consecutive points before the peak, and decrease
or remain constant for two consecutive points after the peak.

The 7-step algorithm:
(1) finds GRID points indicating meal starts
(2) identifies modified GRID points after minimum duration
(3) locates maximum glucose within the subsequent time window
(4) detects all local maxima using the two-consecutive-point criteria
(5) refines peaks from local maxima candidates
(6) maps GRID points to peaks within 4-hour constraint
(7) redistributes overlapping peaks.
\end{Description}
%
\begin{Usage}
\begin{verbatim}
maxima_grid(df, threshold = 130, gap = 60, hours = 2)
\end{verbatim}
\end{Usage}
%
\begin{Arguments}
\begin{ldescription}
\item[\code{df}] A dataframe containing continuous glucose monitoring (CGM) data.
Must include columns:
\begin{itemize}

\item{} \code{id}: Subject identifier (string or factor)
\item{} \code{time}: Time of measurement (POSIXct)
\item{} \code{gl}: Glucose value (integer or numeric, mg/dL)

\end{itemize}


\item[\code{threshold}] GRID slope threshold in mg/dL/hour for event classification (default: 130)

\item[\code{gap}] Gap threshold in minutes for event detection (default: 60).
This parameter defines the minimum time interval between consecutive GRID events.

\item[\code{hours}] Time window in hours for maxima analysis (default: 2)
\end{ldescription}
\end{Arguments}
%
\begin{Value}
A list containing:
\begin{itemize}

\item{} \code{results}: Tibble with combined maxima and GRID analysis results, with columns:
\begin{itemize}

\item{} \code{id}: Subject identifier
\item{} \code{grid\_time}: Timestamp of GRID event detection (POSIXct)
\item{} \code{grid\_gl}: Glucose value at GRID event (mg/dL)
\item{} \code{maxima\_time}: Timestamp of peak glucose (POSIXct)
\item{} \code{maxima\_glucose}: Peak glucose value (mg/dL)
\item{} \code{time\_to\_peak\_min}: Time from GRID event to peak in minutes
\item{} \code{grid\_index}: R-based (1-indexed) row number of GRID event; \code{grid\_time == df\$time[grid\_index]}, \code{grid\_gl == df\$gl[grid\_index]}
\item{} \code{maxima\_index}: R-based (1-indexed) row number of peak; \code{maxima\_time == df\$time[maxima\_index]}, \code{maxima\_glucose == df\$gl[maxima\_index]}

\end{itemize}

\item{} \code{episode\_counts}: Tibble with episode counts per subject (\code{id}, \code{episode\_counts})

\end{itemize}

\end{Value}
%
\begin{Section}{Algorithm (7 steps)}

1) GRID → 2) modified GRID → 3) window maxima → 4) local maxima → 5) refine peaks →
6) map GRID to peaks (<=4h) → 7) redistribute overlapping peaks.
\end{Section}
%
\begin{SeeAlso}
\LinkA{grid}{grid}, \LinkA{mod\_grid}{mod.Rul.grid}, \LinkA{find\_local\_maxima}{find.Rul.local.Rul.maxima}, \LinkA{find\_new\_maxima}{find.Rul.new.Rul.maxima}, \LinkA{transform\_df}{transform.Rul.df}

Other GRID pipeline: 
\code{\LinkA{detect\_between\_maxima}{detect.Rul.between.Rul.maxima}()},
\code{\LinkA{find\_local\_maxima}{find.Rul.local.Rul.maxima}()},
\code{\LinkA{find\_max\_after\_hours}{find.Rul.max.Rul.after.Rul.hours}()},
\code{\LinkA{find\_max\_before\_hours}{find.Rul.max.Rul.before.Rul.hours}()},
\code{\LinkA{find\_min\_after\_hours}{find.Rul.min.Rul.after.Rul.hours}()},
\code{\LinkA{find\_min\_before\_hours}{find.Rul.min.Rul.before.Rul.hours}()},
\code{\LinkA{find\_new\_maxima}{find.Rul.new.Rul.maxima}()},
\code{\LinkA{grid}{grid}()},
\code{\LinkA{mod\_grid}{mod.Rul.grid}()},
\code{\LinkA{start\_finder}{start.Rul.finder}()},
\code{\LinkA{transform\_df}{transform.Rul.df}()}
\end{SeeAlso}
%
\begin{Examples}
\begin{ExampleCode}
# Load sample data
library(iglu)
data(example_data_5_subject)
data(example_data_hall)

# Combined analysis on smaller dataset
maxima_result <- maxima_grid(example_data_5_subject, threshold = 130, gap = 60, hours = 2)
print(maxima_result$episode_counts)
print(maxima_result$results)

# More sensitive analysis
sensitive_maxima <- maxima_grid(example_data_5_subject, threshold = 120, gap = 30, hours = 1)
print(sensitive_maxima$episode_counts)
print(sensitive_maxima$results)

# Analysis on larger dataset
large_maxima <- maxima_grid(example_data_hall, threshold = 130, gap = 60, hours = 2)
print(large_maxima$episode_counts)
print(large_maxima$results)
\end{ExampleCode}
\end{Examples}
\HeaderA{mod\_grid}{Modified GRID Analysis}{mod.Rul.grid}
\keyword{GRID pipeline}{mod\_grid}
%
\begin{Description}
Constructs a modified GRID series by reapplying the GRID logic with a designated
gap (e.g., 60 minutes) and analysis window in hours (e.g., 2 hours). It
reassigns GRID events under these constraints to produce a modified grid
suitable for downstream maxima mapping and episode analysis.
\end{Description}
%
\begin{Usage}
\begin{verbatim}
mod_grid(df, grid_point_df, hours = 2, gap = 15)
\end{verbatim}
\end{Usage}
%
\begin{Arguments}
\begin{ldescription}
\item[\code{df}] A dataframe containing continuous glucose monitoring (CGM) data.
Must include columns:
\begin{itemize}

\item{} \code{id}: Subject identifier (string or factor)
\item{} \code{time}: Time of measurement (POSIXct)
\item{} \code{gl}: Glucose value (integer or numeric, mg/dL)

\end{itemize}


\item[\code{grid\_point\_df}] A dataframe with column \code{start\_indices} (start points for re-applied GRID)

\item[\code{hours}] Time window in hours for analysis (default: 2)

\item[\code{gap}] Gap threshold in minutes for event detection (default: 15).
This parameter defines the minimum time interval between consecutive GRID events.
\end{ldescription}
\end{Arguments}
%
\begin{Value}
A list containing:
\begin{itemize}

\item{} \code{mod\_grid\_vector}: Tibble with modified GRID results (\code{mod\_grid})
\item{} \code{episode\_counts}: Tibble with episode counts per subject (\code{id}, \code{episode\_counts})
\item{} \code{episode\_start}: Tibble with all episode starts with columns:
\begin{itemize}

\item{} \code{id}: Subject identifier
\item{} \code{time}: Timestamp at which the event occurs; equivalent to \code{df\$time[indices]}
\item{} \code{gl}: Glucose value at the event; equivalent to \code{df\$gl[indices]}
\item{} \code{indices}: R-based (1-indexed) row number(s) in \code{df} denoting where the event occurs

\end{itemize}


\end{itemize}

\end{Value}
%
\begin{Section}{Units and sampling}

- \code{gap} is minutes; \code{hours} is hours; \code{time} is POSIXct.
\end{Section}
%
\begin{SeeAlso}
\LinkA{grid}{grid}, \LinkA{find\_max\_after\_hours}{find.Rul.max.Rul.after.Rul.hours}, \LinkA{find\_new\_maxima}{find.Rul.new.Rul.maxima}

Other GRID pipeline: 
\code{\LinkA{detect\_between\_maxima}{detect.Rul.between.Rul.maxima}()},
\code{\LinkA{find\_local\_maxima}{find.Rul.local.Rul.maxima}()},
\code{\LinkA{find\_max\_after\_hours}{find.Rul.max.Rul.after.Rul.hours}()},
\code{\LinkA{find\_max\_before\_hours}{find.Rul.max.Rul.before.Rul.hours}()},
\code{\LinkA{find\_min\_after\_hours}{find.Rul.min.Rul.after.Rul.hours}()},
\code{\LinkA{find\_min\_before\_hours}{find.Rul.min.Rul.before.Rul.hours}()},
\code{\LinkA{find\_new\_maxima}{find.Rul.new.Rul.maxima}()},
\code{\LinkA{grid}{grid}()},
\code{\LinkA{maxima\_grid}{maxima.Rul.grid}()},
\code{\LinkA{start\_finder}{start.Rul.finder}()},
\code{\LinkA{transform\_df}{transform.Rul.df}()}
\end{SeeAlso}
%
\begin{Examples}
\begin{ExampleCode}
# Load sample data
library(iglu)
data(example_data_5_subject)
data(example_data_hall)

# First, get grid points
grid_result <- grid(example_data_5_subject, gap = 60, threshold = 130)

# Perform modified GRID analysis
mod_result <- mod_grid(example_data_5_subject, grid_result$grid_vector, hours = 2, gap = 60)
print(paste("Modified grid points:", nrow(mod_result$mod_grid_vector)))

# Modified analysis with different parameters
mod_result_1h <- mod_grid(example_data_5_subject, grid_result$grid_vector, hours = 1, gap = 40)

# Analysis on larger dataset
large_grid <- grid(example_data_hall, gap = 60, threshold = 130)
large_mod_result <- mod_grid(example_data_hall, large_grid$grid_vector, hours = 2, gap = 60)
print(paste("Modified grid points in larger dataset:", nrow(large_mod_result$mod_grid_vector)))
\end{ExampleCode}
\end{Examples}
\HeaderA{orderfast}{Fast Ordering Function}{orderfast}
%
\begin{Description}
Orders a dataframe by \code{id} and \code{time} columns efficiently using base R's
\code{order}. Optimized for large CGM datasets, it returns the input with rows
sorted by subject then timestamp while preserving all columns.

Orders a dataframe by id and time columns
\end{Description}
%
\begin{Usage}
\begin{verbatim}
orderfast(df)
\end{verbatim}
\end{Usage}
%
\begin{Arguments}
\begin{ldescription}
\item[\code{df}] A dataframe with 'id' and 'time' columns
\end{ldescription}
\end{Arguments}
%
\begin{Value}
A dataframe ordered by id and time

A dataframe ordered by id and time
\end{Value}
%
\begin{Examples}
\begin{ExampleCode}
# Load sample data
library(iglu)
data(example_data_5_subject)
data(example_data_hall)

# Shuffle without replacement, then order and compare to baseline
set.seed(123)
shuffled <- example_data_5_subject[sample(seq_len(nrow(example_data_5_subject)),
                                          replace = FALSE), ]
baseline <- orderfast(example_data_5_subject)
ordered_shuffled <- orderfast(shuffled)

# Compare results
print(paste("Identical after ordering:", identical(baseline, ordered_shuffled)))
head(baseline[, c("id", "time", "gl")])
head(ordered_shuffled[, c("id", "time", "gl")])

# Order larger dataset
ordered_large <- orderfast(example_data_hall)
print(paste("Ordered", nrow(ordered_large), "rows in larger dataset"))
df <- data.frame(id = c("b", "a", "a"), time = as.POSIXct(
  c("2024-01-01 01:00:00", "2024-01-01 00:00:00", "2024-01-01 01:00:00"), tz = "UTC"
))
orderfast(df)
\end{ExampleCode}
\end{Examples}
\HeaderA{start\_finder}{Find Start Points for Event Analysis}{start.Rul.finder}
\keyword{GRID pipeline}{start\_finder}
%
\begin{Description}
Finds R-based (1-indexed) positions where the value is 1 in an integer vector 
of 0s and 1s, specifically identifying episode start points. This function 
looks for positions where a 1 follows a 0 or is at the beginning of the vector,
which is useful for identifying the start of glycemic events or episodes.
\end{Description}
%
\begin{Usage}
\begin{verbatim}
start_finder(df)
\end{verbatim}
\end{Usage}
%
\begin{Arguments}
\begin{ldescription}
\item[\code{df}] A dataframe with the first column containing an integer vector of 0s and 1s
\end{ldescription}
\end{Arguments}
%
\begin{Value}
A tibble containing start\_indices with R-based (1-indexed) positions where episodes start
Note: These indices refer to positions in the provided input vector/dataframe, not necessarily rows of the original CGM \code{df} unless that vector was derived directly from \code{df} in row order.
\end{Value}
%
\begin{Section}{Notes}

- Returns R-based \code{start\_indices} positions relative to the provided input vector/dataframe.
- If used on vectors derived from a CGM \code{df}, indices map directly to \code{df} rows.
\end{Section}
%
\begin{SeeAlso}
\LinkA{grid}{grid}, \LinkA{mod\_grid}{mod.Rul.grid}

Other GRID pipeline: 
\code{\LinkA{detect\_between\_maxima}{detect.Rul.between.Rul.maxima}()},
\code{\LinkA{find\_local\_maxima}{find.Rul.local.Rul.maxima}()},
\code{\LinkA{find\_max\_after\_hours}{find.Rul.max.Rul.after.Rul.hours}()},
\code{\LinkA{find\_max\_before\_hours}{find.Rul.max.Rul.before.Rul.hours}()},
\code{\LinkA{find\_min\_after\_hours}{find.Rul.min.Rul.after.Rul.hours}()},
\code{\LinkA{find\_min\_before\_hours}{find.Rul.min.Rul.before.Rul.hours}()},
\code{\LinkA{find\_new\_maxima}{find.Rul.new.Rul.maxima}()},
\code{\LinkA{grid}{grid}()},
\code{\LinkA{maxima\_grid}{maxima.Rul.grid}()},
\code{\LinkA{mod\_grid}{mod.Rul.grid}()},
\code{\LinkA{transform\_df}{transform.Rul.df}()}
\end{SeeAlso}
%
\begin{Examples}
\begin{ExampleCode}
# Load sample data
library(iglu)
data(example_data_5_subject)
data(example_data_hall)

# Create a binary vector indicating episode starts
binary_vector <- c(0, 0, 1, 1, 0, 1, 0, 0, 1, 1)
df <- data.frame(episode_starts = binary_vector)

# Find R-based indices where episodes start
start_points <- start_finder(df)
print(paste("Start indices:", paste(start_points$start_indices, collapse = ", ")))

# Use with actual GRID results
grid_result <- grid(example_data_5_subject, gap = 15, threshold = 130)
grid_starts <- start_finder(grid_result$grid_vector)
print(paste("GRID episode starts:", length(grid_starts$start_indices)))

# Analysis on larger dataset
large_grid <- grid(example_data_hall, gap = 15, threshold = 130)
large_starts <- start_finder(large_grid$grid_vector)
print(paste("GRID episode starts in larger dataset:", length(large_starts$start_indices)))
\end{ExampleCode}
\end{Examples}
\HeaderA{transform\_df}{Transform Dataframe for Analysis}{transform.Rul.df}
\keyword{GRID pipeline}{transform\_df}
%
\begin{Description}
Performs data transformations required for GRID analysis, including
mapping GRID episode starts to maxima within a 4-hour window and
merging grid and maxima information. This function prepares data
for downstream analysis by combining these results.
\end{Description}
%
\begin{Usage}
\begin{verbatim}
transform_df(grid_df, maxima_df)
\end{verbatim}
\end{Usage}
%
\begin{Arguments}
\begin{ldescription}
\item[\code{grid\_df}] A dataframe containing grid analysis results

\item[\code{maxima\_df}] A dataframe containing maxima detection results
\end{ldescription}
\end{Arguments}
%
\begin{Value}
A tibble with transformed data containing columns (\code{id}, \code{grid\_time}, \code{grid\_gl}, \code{maxima\_time}, \code{maxima\_gl})
\end{Value}
%
\begin{SeeAlso}
\LinkA{grid}{grid}, \LinkA{find\_new\_maxima}{find.Rul.new.Rul.maxima}, \LinkA{detect\_between\_maxima}{detect.Rul.between.Rul.maxima}

Other GRID pipeline: 
\code{\LinkA{detect\_between\_maxima}{detect.Rul.between.Rul.maxima}()},
\code{\LinkA{find\_local\_maxima}{find.Rul.local.Rul.maxima}()},
\code{\LinkA{find\_max\_after\_hours}{find.Rul.max.Rul.after.Rul.hours}()},
\code{\LinkA{find\_max\_before\_hours}{find.Rul.max.Rul.before.Rul.hours}()},
\code{\LinkA{find\_min\_after\_hours}{find.Rul.min.Rul.after.Rul.hours}()},
\code{\LinkA{find\_min\_before\_hours}{find.Rul.min.Rul.before.Rul.hours}()},
\code{\LinkA{find\_new\_maxima}{find.Rul.new.Rul.maxima}()},
\code{\LinkA{grid}{grid}()},
\code{\LinkA{maxima\_grid}{maxima.Rul.grid}()},
\code{\LinkA{mod\_grid}{mod.Rul.grid}()},
\code{\LinkA{start\_finder}{start.Rul.finder}()}
\end{SeeAlso}
%
\begin{Examples}
\begin{ExampleCode}
# Load sample data
library(iglu)
data(example_data_5_subject)
data(example_data_hall)

# Complete pipeline example with smaller dataset
threshold <- 130
gap <- 60
hours <- 2
# 1) Find GRID points
grid_result <- grid(example_data_5_subject, gap = gap, threshold = threshold)
# 2) Find modified GRID points before 2 hours minimum
mod_grid <- mod_grid(example_data_5_subject, 
                     start_finder(grid_result$grid_vector), 
                     hours = hours, 
                     gap = gap)

# 3) Find maximum point 2 hours after mod_grid point
mod_grid_maxima <- find_max_after_hours(example_data_5_subject, 
                                        start_finder(mod_grid$mod_grid_vector), 
                                        hours = hours)

# 4) Identify local maxima around episodes/windows
local_maxima <- find_local_maxima(example_data_5_subject)

# 5) Among local maxima, find maximum point after two hours
final_maxima <- find_new_maxima(example_data_5_subject, 
                                mod_grid_maxima$max_indices, 
                                local_maxima$local_maxima_vector)

# 6) Map GRID points to maximum points (within 4 hours)
transform_maxima <- transform_df(grid_result$episode_start, final_maxima)

# 7) Redistribute overlapping maxima between GRID points
final_between_maxima <- detect_between_maxima(example_data_5_subject, transform_maxima)
# Complete pipeline example with larger dataset (example_data_hall)
# This demonstrates the same workflow on a more comprehensive dataset
hall_threshold <- 130
hall_gap <- 60
hall_hours <- 2

# 1) Find GRID points on larger dataset
hall_grid_result <- grid(example_data_hall, gap = hall_gap, threshold = hall_threshold)

# 2) Find modified GRID points
hall_mod_grid <- mod_grid(example_data_hall, 
                         start_finder(hall_grid_result$grid_vector), 
                         hours = hall_hours, 
                         gap = hall_gap)

# 3) Find maximum points after mod_grid
hall_mod_grid_maxima <- find_max_after_hours(example_data_hall, 
                                            start_finder(hall_mod_grid$mod_grid_vector), 
                                            hours = hall_hours)

# 4) Identify local maxima
hall_local_maxima <- find_local_maxima(example_data_hall)

# 5) Find new maxima
hall_final_maxima <- find_new_maxima(example_data_hall, 
                                    hall_mod_grid_maxima$max_indices, 
                                    hall_local_maxima$local_maxima_vector)

# 6) Transform data
hall_transform_maxima <- transform_df(hall_grid_result$episode_start, hall_final_maxima)

# 7) Detect between maxima
hall_final_between_maxima <- detect_between_maxima(example_data_hall, hall_transform_maxima)
\end{ExampleCode}
\end{Examples}
\printindex{}
\end{document}
